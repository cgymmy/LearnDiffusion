\section{Semilinear Stochastic PDEs}
\subsection{Semilinear SPDE}
Then we come to the time-dependent SPDE. We study the stochastic semilinear evolution equation:
\begin{equation}
    du = [\Delta u + f(u)]dt + G(u)dW(t, x)
\end{equation}
\begin{definition}[Semilinear SPDE]
    Simmilar to normal time-dependent PDE, we treat SPDE like this as semilinear SODEs on a Hilbert space, like
\begin{equation}
    du = [-Au+f(u)]dt + G(u)dW(t)
\end{equation}
where $-A$ is a linear operator that generates a semigroup $S(t)=e^{-tA}$. 
\end{definition}

\begin{example}[Phase-field model]
\begin{equation}
    du = [\epsilon \Delta u + u - u^3]dt + \sigma dW(t, x)
\end{equation}
\end{example}

\begin{example}[Fluid Flow]
    \begin{equation}
        \begin{aligned}
            &u_t = \epsilon \Delta u - \nabla p - (u\cdot \nabla)u\\
            &\nabla\cdot u = 0
        \end{aligned}
    \end{equation}
\end{example}


So like we deal with integration of stochastic process like Itos or stratonovich, we need to generalize the Brownian Motion by introducing spatial variable to W(t). 
Here we define Q-Wiener Process. 
\subsection{Q wiener process}
First, we assume $U$ is a Hilbert space. And $(\Omega, \mathbf{F}, \mathbf{F}_t, \mathbb{R})$ is a filtered probability space. 
\begin{definition}[Q]\label{Q}
    $Q \in \mathcal{L}(U)$ is non-negative definite and symmetric. 
    Further, Q has an orthonormal basis $\{ \mathcal{X}_j : j \in \mathcal{N}\}$ of eigenfunctions with corresponding eigenvalues $q_j \geq 0$  such that $\sum_{j\in\mathcal{N}} q_j < \infty$ (i.e., Q is of trace class).
\end{definition}

\begin{definition}[Q-Wiener Process]
    A U-valued stochastic process $\{W(t):t\geq 0\}$ is $Q$-Wiener process if 
    \begin{itemize}
        \item W(0) = 0 a.s.
        \item W(t) is a continuous function $\mathbb{R}^+\rightarrow U$, for each $\omega \in \Omega$.
        \item W(t) is $\mathcal{F}_t$-adapted and $W(t) - W(s)$ is independent of $\mathcal{F}_s$ for $s\leq t$
        \item $W(t) - W(s)\sim N(0, (t-s)Q)$ for all $0\leq s\leq t$
    \end{itemize}
\end{definition}

\begin{theorem}[Q-Wiener Process]
    Assume we have $Q$ defined in \ref{Q}. Then, $W(t)$ is a Q-Wiener process if and only if 
    \begin{equation}
        W(t)=\sum_{j=1}^\infty \sqrt{q_j}\mathcal{X}_j\beta_j(t)
    \end{equation}
    which is converges in $L^2\left(\Omega, C([0, T], U)\right)$ and $\beta_j(t)$ are iid $\mathcal{F}_t$-Brownian motions and the series converges in $L^2(\Omega,U)$.
\end{theorem}


\begin{theorem}[$H_{\operatorname{per}}^r(0, a)$-valued process]
    ...
\end{theorem}

\begin{theorem}[$H_0^r(0, a)$-valued process]
    ...
\end{theorem}

So, in place of $L^2(D)$, we develop the theory on a separable Hilbert space U with norm $\|\cdot\|_U$ and inner product $\langle \cdot, \cdot\rangle _U$ and define the Q-Wiener process ${W (t) : t \geq 0}$ as a U-valued process. 

\subsection{Cylindrical Wiener Process}

We mention the important case of Q = I, which is not trace class on an infinite-dimensional space U (as $q_j = 1$ for all j) so that the series does not converge in $L^2(\Omega,U)$ . To extend the definition of a Q-Wiener process, we introduce the cylindrical Wiener process.

The key point is to introduce a second space U1 such that $U\subset U_1$ and Q = I is a trace class operator when extended to $U_1$. 

Then we can define cylindrical Wiener process:  
\begin{definition}[Cylindrical Wiener Process]
Let  U  be a separable Hilbert space. The cylindrical Wiener process (also called space-time white noise) is the  U-valued stochastic process  W(t)  defined by
$$W(t)=\sum_{j=1}^{\infty} \mathcal{X}_{j} \beta_{j}(t)$$
where  $\left\{\mathcal{X}_{j}\right\}$  is any orthonormal basis of  U  and  $\beta_{j}(t)$  are iid  $\mathcal{F}_{t}$-Brownian motions. 
\end{definition}

\begin{theorem}
    If for the second Hilbert space $U_1$, and the inclusion map $\mathcal{I}: U \rightarrow U_1$ is Hilbert-Schmidt. 
    Then, the cylindrical Wiener process is a Q-Wiener process well-defined on $U_1$(Converges in $L^2(U, U_1)$).
\end{theorem}

\subsection{Ito integral solution}
Here we consider the Ito integral $\int_0^t B(s)dW(s)$ for a Q-Wiener process $W(s)$. 
Since $dW_t$ takes value in Hilbert space $U$, and we treat SPDE in Hilbert space $H$, the integral will also take value in Hilbert space $H$.

Hence, $B(s)$ should be $\mathcal{L}_0^2(U_0, H)$-valued process, where $U_0\subset U$ known as Cameron-Martin space. 
So, $B(s)$ is an operator from $U_0$ to $H$. Then, we consider the set of operator $B$.
\begin{definition}[$L_0^2$ space]
    Let $U_0:=\{Q^{\frac{1}{2}}u: u\in U\}$, the set of linear operators $B:U_0\rightarrow H$ is noted as $L_0^2$ s.t. 
    \begin{equation}
        \|B\|_{L_0^2} := \left(\sum_{j=1}^\infty \|BQ^{\frac{1}{2}}\mathcal{X}_j\|^2\right)^{\frac{1}{2}} = \|BQ^{\frac{1}{2}}\|_{\operatorname{HS}(U_0, H)}<\infty
    \end{equation}
\end{definition}
\begin{remark}
    If $G$ is invertible, $L_0^2$ is the space of Hilbert-Schmidt operators $\operatorname{HS}(U_0, H)$.
\end{remark}

\begin{definition}
    The stochastic integral can be defined by
    \begin{equation}
    \int_0^t B(s)dW(s) := \sum_{j=1}^\infty \int_0^t B(s)\sqrt{q_j}\mathcal{X}_j d\beta_j(s)
\end{equation}
So, we can have the truncated form:
\begin{equation}
    \int_0^t B(s)dW^J(s) = \sum_{j=1}^J \int_0^t B(s)\sqrt{q_j}\mathcal{X}_j d\beta_j(s)
\end{equation}
\end{definition}


\subsection{Semilinear SPDE}
Consider the semilinear SPDE:
\begin{equation}
    du = [-Au+f(u)]dt + G(u)dW(t)
\end{equation}
given the initial condition $u_0\in H$ and $A:\mathcal{D}\subset H\rightarrow H$ is a linear operator, $f: H\rightarrow H$ and $G: H\rightarrow L_0^2$.
\begin{example}
    Consider the stochastic heat equation:
    \begin{equation}
        du = \Delta u dt + \sigma dW(t, x), u(0, x) = u_0(x)\in L^2(D)
    \end{equation}
    where $D$ is a bounded domain in $\mathbb{R}^d$ and $\sigma$ is a constant. 
    Also, homogeneous Dirichlet boundary condition is imposed on D. Hence,
    \begin{equation}
        H = U = L^2(D), f(u) = 0, G(u) = \sigma I
    \end{equation}
    We see that $A = -\Delta$ with domain $\mathcal{D}(A) = H^2(D)\cap H_0^1(D)$.
\end{example}
In the deterministic setting of PDEs, there are a number of different concepts of solution. Here is the same for SPDEs.
We can also define strong solution, weak solution and mild solution.
\begin{definition}[strong solution]
    A predictable  H -valued process  $\{u(t): t \in[0, T]\}$  is called a strong solution if
\begin{equation}
u(t)=u_{0}+\int_{0}^{t}[-A u(s)+f(u(s))] d s+\int_{0}^{t} G(u(s)) d W(s), \quad \forall t \in[0, T] 
\end{equation}
\end{definition}

\begin{definition}[weak solution]
    A predictable  H -valued process  $\{u(t): t \in[0, T]\}$  is called a weak solution if
\begin{equation}
\langle u(t), v\rangle=  \left\langle u_{0}, v\right\rangle+\int_{0}^{t}[-\langle u(s), A v\rangle+\langle f(u(s)), v\rangle] d s +\int_{0}^{t}\langle G(u(s)) d W(s), v\rangle, \quad \forall t \in[0, T], v \in \mathcal{D}(A)
\end{equation}
where
$$\int_{0}^{t}\langle G(u(s)) d W(s), v\rangle:=\sum_{j=1}^{\infty} \int_{0}^{t}\left\langle G(u(s))\sqrt{q_{j}} \mathcal{X}_{j}, v\right\rangle d \beta_{j}(s) .$$
\end{definition}

\begin{definition}[mild solution]
A predictable  H -valued process  $\{u(t): t \in[0, T]\}$  is called a mild solution if for  $t \in[0, T] $

$$u(t)=\mathrm{e}^{-t A} u_{0}+\int_{0}^{t} \mathrm{e}^{-(t-s) A} f(u(s)) d s+\int_{0}^{t} \mathrm{e}^{-(t-s) A} G(u(s)) d W(s),$$
where  $\mathrm{e}^{-t A}$  is the semigroup generated by  $-A$. The right hand side is also called stochastic convolution.
\end{definition}

\begin{example}[stochastic heat equation in one dimension]
    Consider the weak solution of 1D heat SPDE with  $D=(0, \pi)$, so that  $-A$  has eigenfunctions  $\phi_{j}(x)=\sqrt{2 / \pi} \sin (j x)$ 
    and eigenvalues  $\lambda_{j}=j^{2}$  for  $j \in \mathbb{N}$. Suppose that  $W(t)$  is a  Q -Wiener process 
    and the eigenfunctions  $\mathcal{X}_{j}$ of Q are the same as the eigenfunctions  $\phi_{j}$  of  A. A weak solution satisfies: $\forall v \in \mathcal{D}(A)$, 

\begin{equation}
    \begin{aligned}
    \langle u(t), v\rangle_{L^{2}(0, \pi)}= & \left\langle u_{0}, v\right\rangle_{L^{2}(0, \pi)}+\int_{0}^{t}\langle-u(s), A v\rangle_{L^{2}(0, \pi)} d s \\
    & +\sum_{j=1}^{\infty} \int_{0}^{t} \sigma \sqrt{q_{j}}\left\langle\phi_{j}, v\right\rangle_{L^{2}(0, \pi)} d \beta_{j}(s)
    \end{aligned}
\end{equation}
Assume $u(t)=\sum_{j=1}^{\infty} \hat{u}_{j}(t) \phi_{j}$  for  $\hat{u}_{j}(t):=\left\langle u(t), \phi_{j}\right\rangle_{L^{2}(0, \pi)}$ . Take  $v=\phi_{j}$ , we have 
\begin{equation}
    \hat{u}_{j}(t)=\hat{u}_{j}(0)+\int_{0}^{t}\left(-\lambda_{j}\right) \hat{u}_{j}(s) d s+\int_{0}^{t} \sigma \sqrt{q}_{j} d \beta_{j}(s) .
\end{equation}
Hence,  $\hat{u}_{j}(t)$  satisfies the SODE
\begin{equation}
d \hat{u}_{j}=-\lambda_{j} \hat{u}_{j} d t+\sigma \sqrt{q_{j}} d \beta_{j}(t)
\end{equation}
Therefore, each coefficient  $\hat{u}_{j}(t)$  is an Ornstein-Uhlenbeck (OU) process (see Examples 8.1 and 8.21), which is a Gaussian process with variance
\begin{equation}
\operatorname{Var}\left(\hat{u}_{j}(t)\right)=\frac{\sigma^{2} q_{j}}{2 \lambda_{j}}\left(1-\mathrm{e}^{-2 \lambda_{j} t}\right)
\end{equation}

For initial data  $u_{0}=0$ , we obtain, by the Parseval identity (1.43),
\begin{equation}
\|u(t)\|_{L^{2}\left(\Omega, L^{2}(0, \pi)\right)}^{2}=\mathbb{E}\left[\sum_{j=1}^{\infty}\left|\hat{u}_{j}(t)\right|^{2}\right]=\sum_{j=1}^{\infty} \frac{\sigma^{2} q_{j}}{2 \lambda_{j}}\left(1-\mathrm{e}^{-2 \lambda_{j} t}\right) .
\end{equation}
\end{example}
