\section{Introduction to SDEs}
\subsection{SODEs}
\begin{problem}
    Assume we have a Stochastic Differential Equation like:
    \begin{equation}\label{sde}
        dX_t = f(X_t, t)dt + G(X_t, t)dW_t
    \end{equation}
    where $X_t\in \mathbf{R}^d,f\in \mathcal{L}(\mathbf{R}^{d+1}, \mathbf{R}^d)$, and $W_t$ is m-dim Brownian Motion with diffusion matrix $Q$, 
    $G(X_t, t)\in \mathcal{L}(\mathbf{R}^{m+1}, \mathbf{R}^d)$, with initial condition $X_0\sim p(X_0)$.
\end{problem}
\subsection{The It'so and Stratonovich Stochastic Integrals}
\subsection{Ito's Formula}

\subsection{Mean and Covariance}
We can derive the mean and covariance of SDE.
By applying Ito's formula to $\phi(x, t)$, then
\begin{equation}
    \frac{d E[\phi]}{d t}=E\left[\frac{\partial \phi}{\partial t}\right]+\sum_{i} E\left[\frac{\partial \phi}{\partial x_{i}} f_{i}(X_t, t)\right]+\frac{1}{2} \sum_{i j} E\left[\frac{\partial^{2} \phi}{\partial x_{i} \partial x_{j}}\left[G Q G^{\top}\right]_{i j}\right]
\end{equation}
By taking $\phi(X, t)=x_i$ and $\phi(X, t)=x_ix_j-m(t)_im(t)_j$, we have the mean function $m(t)=E[X_t]$ 
and covariance function $c(t)=E\left[\left(X_t-m(t)\right)\left(X_t-m(t)\right)^T\right]$ respectively, s.t.
\begin{equation}\label{SDEMC}
    \left\{
        \begin{aligned}
            &\frac{d m}{d t}=E\left[f(X_t, t)\right]\\
            &\frac{d c}{d t}=E\left[f(X, t)(X-m(t)^T)\right]+E\left[(X-m(t)f^T(X, t))\right]+E\left[G(X_t, t)QG^T(X_t, t)\right]
        \end{aligned}
    \right.
\end{equation}
So we can estimate the mean and covariance of solution to SDE. However, these equations cannot be used as such, 
because only in the Gaussian case do the expectation and covariance actually characterize the distribution. 

The linear SDE has explicit solution. Assume the linear SDe has the form 
\begin{equation}
    dX_t =\left(K(t)X_t + B(t)\right)dt + G(t)dW_t
\end{equation}
where $K(t)\in \mathbf{R}^{d\times d}, B(t)\in \mathbf{R}^{d}, G(t)\in \mathbf{R}^{d\times m}$ are given functions. 
$X_t \in \mathbf{R}^d$ is the state vector, $W_t \in \mathbf{R}^m$ is the Brownian Motion with diffusion matrix $Q$.

\begin{theorem}
    The explicit solution to the linear SDE is given by:
    \begin{equation}\label{explicitsolLSDE}
        X_t = \Psi(t, t_0)X_0 + \int_{t_0}^t \Psi(t, s)B(s)ds + \int_{t_0}^t \Psi(t, s)G(s)dW_s
    \end{equation}
    where $\Psi(t, t_0)$ is the transition matrix of the linear SDE, which satisfies the following matrix ODE:
    \begin{equation}
        \frac{d\Psi}{dt} = K(t)\Psi(t, t_0), \Psi(t_0, t_0) = I
    \end{equation}
    Hence, $X_t$ is a Gaussian process(A linear transformation of Brownian Motion which is a Gaussian process).
\end{theorem}
\begin{proof}
    Multiply both sides of the SDE by Integrating factor $\Psi(t_0, t)$ and apply Ito's formula to $\Psi(t_0, t)X_t$.

    See Sarkka P49.
\end{proof}
As discussed above, we can compute the mean and covariance function of solution to linear SDE. 
\begin{theorem}
    The mean and covariance function of solution to linear SDE are given by:
    \begin{equation}
        \left\{
            \begin{aligned}
                &\frac{d m}{d t} = K(t)m(t) + B(t)\\
                &\frac{d c}{d t} = K(t)c(t) + c(t)K^T(t)+ G(t)QG^T(t)
            \end{aligned}
        \right.
    \end{equation}
    with initial condition $m_0 =m(t_0)=E[X_0], c_0 =c(t_0)=Cov(X_0)$. Then the solution is given by solving the above ODEs:
    \begin{equation}\label{LSDEMC}
        \left\{
            \begin{aligned}
                &m(t) = \Psi(t, t_0)m_0 + \int_{t_0}^t \Psi(t, s)B(s)ds\\
                &c(t) = \Psi(t, t_0)c_0\Psi^T(t, t_0) + \int_{t_0}^t \Psi(t, s)G(s)QG^T(s)\Psi^T(t, s)ds
            \end{aligned}
        \right.
    \end{equation}

\end{theorem}
\begin{proof}
    Apply $F(X, t)=K(t)X + B(t), G(X, t)=G(t)$ to \ref{SDEMC}.
\end{proof}

Hence the solution to linear SDE is a Gaussian process with mean and covariance function given by the above ODEs.
\begin{theorem}
The solution to LSDE is Gaussian:
\begin{equation}
    p(X, t) = \mathcal{N}(X(t)|m(t), c(t))
\end{equation}
Specially when $X_0 = x_0$ is fixed, then 
\begin{equation}\label{transitiondensity}
    p(X,t|X_0=x_0) = \mathcal{N}(X(t)|m(t|x_0), c(t|x_0))
\end{equation}
That is, $m_0 = x_0, c_0 = 0$. Then we have:
\begin{equation}
    \left\{
        \begin{aligned}
            &m(t|x_0) = \Psi(t, t_0)x_0 + \int_{t_0}^t \Psi(t, s)B(s)ds\\
            &c(t|x_0) = \int_{t_0}^t \Psi(t, s)G(s)QG^T(s)\Psi^T(t, s)ds
        \end{aligned}
    \right.
\end{equation}
\end{theorem}
\begin{proof}
    The proof is straight foward either by applying $m_0 = x_0, c_0 = 0$ to \ref{LSDEMC} or by eq \ref{explicitsolLSDE}.
\end{proof}
So, to sum up, linear SDE has great properties! The distribution is completedly decided by the inital condition.
Also, if we generate $X_0$ to $X_{t_k}$, which means that we begin SDE at $t_i$ with $X_{t_i}$, we have the equivalent discretization of SDE:
\begin{theorem}
    Original SDE is weakly, in distribution, equivalent to the following discrete-time SDE:
    \begin{equation}\label{DTSDE}
        X_{t_{i+1}} = A_iX_{t_i} + B_i + G_i
    \end{equation} 
    where
    \begin{equation}\left\{
        \begin{aligned}
            A_i &= \Psi(t_{i+1}, t_i)\\
            B_i &= \int_{t_i}^{t_{i+1}} \Psi(t_{i+1}, s)B(s)ds\\
            G_i &= \int_{t_i}^{t_{i+1}} \Psi(t_{i+1}, s)G(s)QG^T(s)\Psi^T(t_{i+1}, s)ds
        \end{aligned}\right.
    \end{equation}
\end{theorem}
\begin{proof}
    The proof is straight forward.
\end{proof}
\begin{theorem}
    The covariance of $X_t$ and $X_s(s<t)$ is given by:
    \begin{equation}
        Cov(X_t, X_s) = \Psi(t, s)c(s)
    \end{equation}
\end{theorem}

\begin{proof}
    See Sarkka P88-89.
\end{proof}



