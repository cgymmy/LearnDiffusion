\newpage
\appendix

\section{Conservation Laws}
\subsection{Flow Map}
\begin{theorem}Two important theorems in calculus:

    1.\textbf{Divergence Theorem}:
    \begin{equation}
        \int_{\Omega} \nabla \cdot \mathbf{F} d x = \int_{\partial \Omega} \mathbf{F} \cdot \mathbf{n} d S
    \end{equation}

    2.\textbf{Reynolds Transport Theorem}:
    \begin{equation}
        \frac{d}{dt}\int_{\Omega(t)} f(t, x) d x = \int_{\Omega (t)} \frac{\partial f}{\partial t} d x + \int_{\partial \Omega (t)} f(t, x) \mathbf{v} \cdot \mathbf{n} d S
    \end{equation}
    where $u$ is the velocity at $\partial \Omega (t)$.
\end{theorem}
Here the $\Omega(t)$ is the domain of the flow, and the $\partial \Omega(t)$ is the boundary of the flow, which is described by the flow map $\phi_s^t$. Here is the definition.
\begin{definition}[Flow Map]
    Assume a description of some characteristic of particle $\mathbf{P}$, like the position or the boundary, as $\mathbf{x}\in \mathcal{R}^m$, then we have a flow map $\phi_s^t(\mathbf{x})\in \mathcal{R}^m$, 
    which means that the flow transimits the characteristic(position) $\mathbf{x}$ from $\mathbf{x}$ at $s$ to $\phi_s^t(\mathbf{x})$ at $t$, controlled by the vector field(velocity field) $\mathbf{F}: \mathcal{R}^m\times \mathcal{R}\to \mathcal{R}^m$:
    \begin{equation}\left\{
        \begin{aligned}
            \frac{d\phi_s^t(\mathbf{x})}{dt} &= \mathbf{F}(\phi_s^t(\mathbf{x}), t)\\
            \phi_s^s(\mathbf{x}) &= \mathbf{x}
        \end{aligned}\right.
    \end{equation}
\end{definition}
\subsubsection{Conservation Laws}
If we assume $\Omega(t)$ is composed of particles, i.e. $\Omega(t)=\phi_{t_0}^t(\Omega)$(when $t = t_0$, $\Omega(t_0)=\Omega$), 
then we by \textbf{conservation of mass}, we have the following theorem:
\begin{theorem}[Continuity Equation]
    By conservation of mass, i.e. $\int_{\Omega(t)} \rho(t, \mathbf{x}) d\mathbf{x} = C$, we have:
    \begin{equation}
        \begin{aligned}
            \frac{d}{dt}\int_{\Omega(t)} \rho(t, \mathbf{x}) d\mathbf{x} &= \int_{\Omega(t)} \frac{\partial \rho}{\partial t} d\mathbf{x} + \int_{\partial \Omega(t)} \rho(t, \mathbf{x}) \mathbf{u} \cdot \mathbf{n} d S\\ 
            &= \int_{\Omega(t)} \left(\frac{\partial \rho}{\partial t}  + \nabla\cdot \left(\rho\mathbf{u}\right)\right) d\mathbf{x}=0
        \end{aligned}
    \end{equation}
    Therefore:
    \begin{equation}
        \frac{\partial \rho}{\partial t}  + \nabla\cdot \left(\rho\mathbf{u}\right) = 0
    \end{equation}
    which is also called \textbf{continuity equation}.
\end{theorem}
\begin{theorem}[Conservation of Momentum]
    By conservation of momentum, i.e. 
    \begin{equation}
        \frac{d}{dt}\int_{\Omega(t)} \rho(t, \mathbf{x})\mathbf{v}(t, \mathbf{x}) d\mathbf{x} = -\int_{\partial \Omega(t)} p\cdot \mathbf{n} d S
    \end{equation}
    we have:
    \begin{equation}
        \frac{\partial (\rho \mathbf{v})}{\partial t} + \nabla\cdot \left(\rho \mathbf{v} \otimes \mathbf{v} + p\right) = 0
    \end{equation}
    where $p$ is the pressure.
\end{theorem}
\begin{theorem}[Conservation of Energy]
    \begin{equation}
        \frac{\partial E}{\partial t} + \nabla\cdot \left(\mathbf{v}(E + p)\right) = 0
    \end{equation}
\end{theorem}
Then we have can get the Euler's equation:
\begin{theorem}[Euler's Equation]
    The Euler's equation is given by:
    \begin{equation}
        \frac{\partial}{\partial t}\begin{bmatrix}
        \rho\\ \rho \mathbf{v}\\E
    \end{bmatrix}
    + \nabla\cdot \begin{bmatrix}
        \rho \mathbf{v}\\ \rho \mathbf{v}\otimes \mathbf{v} + p\\ \mathbf{v}(E + p)
    \end{bmatrix} = 0
\end{equation}
So the general form of conservation laws is given by: suppose $U\in \mathcal{R}^d$ is the conserved quantity, $F$ is $\mathcal{R}^d\to \mathcal{R}^d$ is the flux, 
then we have:
\begin{equation}
    \frac{\partial U}{\partial t} + \nabla\cdot \left(F(U)\right) = 0
\end{equation}

\end{theorem}

