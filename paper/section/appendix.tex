\newpage
\appendix

\section{Conservation Laws}
\begin{theorem}Two important theorems in calculus:

    1.\textbf{Divergence Theorem}:
    \begin{equation}
        \int_{\Omega} \nabla \cdot \mathbf{F} d x = \int_{\partial \Omega} \mathbf{F} \cdot \mathbf{n} d S
    \end{equation}

    2.\textbf{Reynolds Transport Theorem}:
    \begin{equation}
        \frac{d}{dt}\int_{\Omega(t)} f(t, x) d x = \int_{\Omega (t)} \frac{\partial f}{\partial t} d x + \int_{\partial \Omega (t)} f(t, x) \mathbf{v} \cdot \mathbf{n} d S
    \end{equation}
    where $u$ is the velocity at $\partial \Omega (t)$.
\end{theorem}
Here the $\Omega(t)$ is the domain of the flow, and the $\partial \Omega(t)$ is the boundary of the flow, which is described by the flow map $\phi_s^t$. Here is the definition.
\begin{definition}[Flow Map]
    Assume a description of some characteristic of particle $\mathbf{P}$, like the position or the boundary, as $\mathbf{x}\in \mathcal{R}^m$, then we have a flow map $\phi_s^t(\mathbf{x})\in \mathcal{R}^m$, 
    which means that the flow transimits the characteristic(position) $\mathbf{x}$ from $\mathbf{x}$ at $s$ to $\phi_s^t(\mathbf{x})$ at $t$, controlled by the vector field(velocity field) $\mathbf{F}: \mathcal{R}^m\times \mathcal{R}\to \mathcal{R}^m$:
    \begin{equation}\left\{
        \begin{aligned}
            \frac{d\phi_s^t(\mathbf{x})}{dt} &= \mathbf{F}(\phi_s^t(\mathbf{x}), t)\\
            \phi_s^s(\mathbf{x}) &= \mathbf{x}
        \end{aligned}\right.
    \end{equation}
\end{definition}
If we assume $\Omega(t)$ is composed of particles, i.e. $\Omega(t)=\phi_{t_0}^t(\Omega)$(when $t = t_0$, $\Omega(t_0)=\Omega$), 
then we by \textbf{conservation of mass}, we have the following theorem:
\begin{theorem}[Continuity Equation]
    By conservation of mass, i.e. $\int_{\Omega(t)} \rho(t, \mathbf{x}) d\mathbf{x} = C$, we have:
    \begin{equation}
        \begin{aligned}
            \frac{d}{dt}\int_{\Omega(t)} \rho(t, \mathbf{x}) d\mathbf{x} &= \int_{\Omega(t)} \frac{\partial \rho}{\partial t} d\mathbf{x} + \int_{\partial \Omega(t)} \rho(t, \mathbf{x}) \mathbf{u} \cdot \mathbf{n} d S\\ 
            &= \int_{\Omega(t)} \left(\frac{\partial \rho}{\partial t}  + \nabla\cdot \left(\rho\mathbf{u}\right)\right) d\mathbf{x}=0
        \end{aligned}
    \end{equation}
    Therefore:
    \begin{equation}
        \frac{\partial \rho}{\partial t}  + \nabla\cdot \left(\rho\mathbf{u}\right) = 0
    \end{equation}
    which is also called \textbf{continuity equation}.
\end{theorem}
\begin{theorem}[Conservation of Momentum]
    By conservation of momentum, i.e. 
    \begin{equation}
        \frac{d}{dt}\int_{\Omega(t)} \rho(t, \mathbf{x})\mathbf{v}(t, \mathbf{x}) d\mathbf{x} = -\int_{\partial \Omega(t)} p\cdot \mathbf{n} d S
    \end{equation}
    we have:
    \begin{equation}
        \frac{\partial (\rho \mathbf{v})}{\partial t} + \nabla\cdot \left(\rho \mathbf{v} \otimes \mathbf{v} + p\right) = 0
    \end{equation}
    where $p$ is the pressure.
\end{theorem}
\begin{theorem}[Conservation of Energy]
    \begin{equation}
        \frac{\partial E}{\partial t} + \nabla\cdot \left(\mathbf{v}(E + p)\right) = 0
    \end{equation}
\end{theorem}
Then we have can get the Euler's equation:
\begin{theorem}[Euler's Equation]
    The Euler's equation is given by:
    \begin{equation}
        \frac{\partial}{\partial t}\begin{bmatrix}
        \rho\\ \rho \mathbf{v}\\E
    \end{bmatrix}
    + \nabla\cdot \begin{bmatrix}
        \rho \mathbf{v}\\ \rho \mathbf{v}\otimes \mathbf{v} + p\\ \mathbf{v}(E + p)
    \end{bmatrix} = 0
\end{equation}
So the general form of conservation laws is given by: suppose $U\in \mathcal{R}^d$ is the conserved quantity, $F$ is $\mathcal{R}^d\to \mathcal{R}^d$ is the flux, 
then we have:
\begin{equation}
    \frac{\partial U}{\partial t} + \nabla\cdot \left(F(U)\right) = 0
\end{equation}

\section{Linear Algebra}
\begin{definition}[Matrix Kronecker Product]
    Let $A\in \mathbb{R}^{m\times n}$ and $B\in \mathbb{R}^{p\times q}$ be two matrices. The Kronecker product of $A$ and $B$ is defined as:
    \begin{equation}
        A\otimes B = \begin{bmatrix}
            a_{11}B & a_{12}B & \cdots & a_{1n}B\\
            a_{21}B & a_{22}B & \cdots & a_{2n}B\\
            \vdots & \vdots & \ddots & \vdots\\
            a_{m1}B & a_{m2}B & \cdots & a_{mn}B
        \end{bmatrix}
    \end{equation}
\end{definition}

\begin{definition}[Fourier Matrix]
    Let $n$ be a positive integer. The Fourier matrix $W_n$ is defined as:
    \begin{equation}
        W_n = \frac{1}{\sqrt{n}}\begin{bmatrix}
            1 & 1 & \cdots & 1\\
            1 & w & \cdots & w^{n-1}\\
            \vdots & \vdots & \ddots & \vdots\\
            1 & w^{n-1} & \cdots & 1
        \end{bmatrix}
    \end{equation}
\end{definition}

\begin{definition}[Two dimensional Fourier Matrix]
    Let $W_1, W_2$ be $n_1\times n_1$ and $n_2\times n_2$ Fourier matrices respectively. 
    Then the two dimensional Fourier matrix is defined as:
    \begin{equation}
        W = W_2\otimes W_1
    \end{equation}
    which is the $n_1n_2\times n_1n_2$ matrix.
\end{definition}

\begin{theorem}[Two-dimensional DFT]
    The two-dimentional DFT of $V\in \mathbb{C}^{n_1\times n_2}$ is $\hat{V}\in \mathbb{C}^{n_1\times n_2}$ whose elements are given by:
    \begin{equation}
        \hat{V}_{ij} = \sum_{k=0}^{n_1-1}\left(\sum_{l=0}^{n_2-1}V_{kl}w_1^{ik}\right)w_2^{jl}, w_1 = e^{-\frac{2\pi i}{n_1}}, w_2 = e^{-\frac{2\pi i}{n_2}}
    \end{equation}
    where $i = 0, \cdots, n_1 - 1, j = 0, \cdots, n_2-1$. 
    \begin{equation}
        \hat{V} = \sqrt{n_1n_2}\operatorname{array}\left((W_2\otimes W_1)\tilde{v}\right),\ \tilde{v} = vec(V)
    \end{equation}
    Do DFT on column and row respectively.
\end{theorem}

\begin{theorem}[Two-dimensional IDFT]
    The two-dimentional IDFT of $\hat{V}\in \mathbb{C}^{n_1\times n_2}$ is $V\in \mathbb{C}^{n_1\times n_2}$ whose elements are given by:
    \begin{equation}
        V_{kl} = \frac{1}{n_1n_2}\sum_{i=0}^{n_1-1}\sum_{j=0}^{n_2-1}\hat{V}_{ij}w_1^{-ik}w_2^{-jl}, w_1 = e^{\frac{2\pi i}{n_1}}, w_2 = e^{\frac{2\pi i}{n_2}}
    \end{equation}
    where $k = 0, \cdots, n_1-1, l = 0, \cdots, n_2-1$. 
    \begin{equation}
        V = \frac{1}{\sqrt{n_1n_2}}\operatorname{array}\left((W_2\otimes W_1)^*\hat{v}\right),\ \hat{v} = vec(\hat{V})
    \end{equation}
    Do IDFT on column and row respectively.
\end{theorem}

\begin{theorem}[Fourier representation of BCCB matrix]
    Let $C\in \mathbb{R}^{n_1\times n_2}$ be a BCCB matrix, then $C = WDW^*$, where $W$ is the two-dimensional Fourier matrix, 
    and $D$ is a diagonal matrix with diagonal $d = vec(\Lambda)$, where
    \begin{equation}
        \Lambda = \sqrt{n_1n_2}\operatorname{array}\left((W)^*c_{red}\right),\ c_{red}=\operatorname{vec}(C_{red})
    \end{equation}
    
\end{theorem}

\section{Priori}
\subsection{Hilbert space-valued random variable}
\begin{definition}[$L^p(\Omega, H)$ space]
    Let $(\Omega, \mathcal{F}, \mathbb{P})$ be a probability space and $H$ is a Hilbert space with norm $\|\cdot\|$. Then $\mathcal{L}^p(\Omega, H)$ with $1\leq p<\infty$ is the space
    of H-valued $\mathcal{F}$-measurable random vaiables $X:\Omega\rightarrow H$ with $\mathbf{E}[\|X\|^p]<\infty$ and a Banach space with norm:
    \begin{equation}
        \|X\|_{\mathcal{L}^p(\Omega, H)}:=\left(\int_\Omega \|X(\omega)\|^pdP(\omega)\right)^{\frac{1}{p}}=\mathbf{E}[\|X\|^p]^{\frac{1}{p}}
    \end{equation}
\end{definition}
Then we can define the inner product: 
\begin{equation}
    \langle X, Y\rangle_{\mathcal{L}^2(\Omega, H)}:=\int_\Omega \langle X(\omega), Y(\omega)\rangle dP(\omega)
\end{equation}
\begin{definition}[uncorrelated, covariance operator]
    Let $H$ be a Hilbert space. A linear operator $\mathcal{C}:H\rightarrow H$ is the covariance of $H$-valued random variables $X$ and $Y$ if 
    \begin{equation}
        \langle\mathcal{C}\phi, \psi\rangle = \operatorname{Cov}\left(\langle X, \phi\rangle, \langle Y, \psi\rangle\right), \forall \phi, \psi \in H
    \end{equation}
    specially, we show that in finite dimensional case, the covariance matrix conincides with the covariance operator. 
    when $H = \mathbb{R}^d$,
    \begin{equation}
        \begin{aligned}
            &\operatorname{Cov}\left(\langle X, \phi\rangle, \langle Y, \psi\rangle\right) 
            = \operatorname{Cov}\left(\phi^T X, \psi^T Y\right)\\
            =&\mathbf{E}\left[\phi^T(X-\mu_X)(Y-\mu)^T\psi\right] 
            = \phi^T\mathbf{E}\left[(X-\mu_X)(Y-\mu)^T\right]\psi\\
            =&\langle C\phi, \psi\rangle
        \end{aligned}
    \end{equation}
\end{definition}

\begin{definition}[H-valued Gaussian random variable]
    Let $H$ be a Hilbert space. An H-valued random variable $X$ is Gaussian if 
    $\langle X, \phi\rangle$ is a real-valued Gaussian random variable for all $\phi \in H$.
\end{definition}
\subsection{Hilbert-Schmidt operator}
\begin{definition}[Hilbert-Schmidt operator]
    Let $U, H$ be two separable Hilbert spaces with norms $\|\cdot\|, \|\cdot\|_U$ respectively. 
    For an orthonormal basis $\{\phi_j\}$ of $U$, define the Hilbert-Schmidt norm:
    \begin{equation}
        \|L\|_{\operatorname{HS}(U, H)}:=\left(\sum_{j=1}^\infty \|L\phi_j\|_H^2\right)^{\frac{1}{2}}
    \end{equation}
    where $\operatorname{HS}(U, H): = \{L\in \mathcal{L}(U, H): \|L\|_{\operatorname{HS}(U, H)}<\infty\}$ is a Banach space with Hilbert-Schmidt norm.
    And $L\in \operatorname{HS}(U, H)$ is called Hilbert-Schmidt operator.
\end{definition}
\begin{definition}[Integral operator with kernel G]
    For a domain $D$ and a kernel $G\in L^2(D\times D)$, define the integral operator $L$ by
    \begin{equation}
        (Lu)(x) = \int_D G(x, y)u(y)dy, x\in D, u\in L^2(D)
    \end{equation}
    Furthermore, $L$ is a Hilbert-Schmidt operator.
\end{definition}

\subsection{Operator theory}
\begin{theorem}[Sobolev embedding theorem]
    1.Let $W^{r, p}\left(\mathbf{R}^{n}\right)$. Here k is a non-negative integer and $1 \leq p<\infty$. 
    If $k>\ell, p<n$ and $1 \leq p<q<\infty$ are two real numbers such that
    $\frac{1}{p}-\frac{r}{n}=\frac{1}{q}-\frac{\ell}{n}$, then
    \begin{equation}
        W^{r, p}\left(\mathbf{R}^{n}\right) \subseteq W^{\ell, q}\left(\mathbf{R}^{n}\right)
    \end{equation}
    Specially, if $\ell = 0$, then $\frac{1}{p}-\frac{r}{n}=\frac{1}{q}$, then $W^{r, p}\left(\mathbf{R}^{n}\right) \subseteq L^q\left(\mathbf{R}^{n}\right)$.

    2.If $n<pr$ and $\frac{1}{p}-\frac{r}{n}=-\frac{s+\alpha}{n}$, then $W^{r, p}\left(\mathbf{R}^{n}\right) \subseteq C^{s,\alpha}\left(\mathbf{R}^{n}\right)$.

\end{theorem}

\begin{definition}[domain of operator]
    For a linear operator $A:\mathcal{D}(A)\subset H\rightarrow H$, the domain of $A$ is defined as $\mathcal{D}(A)$
\end{definition}
\begin{theorem}[Dirichlet Boundary Condition]
    Consider the Dirichlet problem for Possion equation: for $f\in L^2(0, 1)$, find $u\in H^2(0, 1)$ s.t.
    \begin{equation}
        \begin{aligned}
            &u_{xx} = f, \quad x\in (0, 1)\\
            &u(0) = u(1) = 0
        \end{aligned}
    \end{equation}
    We also assume $u\in H^1_0(0, 1)$. By Sobolev embedding theorem, $u\in H^1_0(0, 1)\subset C([0, 1])$. 
    Then, Laplacian with Dirichlet conditions can be defined as:
    \begin{equation}
        Au:=-u_{xx}, u\in \mathcal{D}(A)=H^2(0,1)\cap H_0^1(0,1)
    \end{equation}
\end{theorem}


\begin{definition}[Periodic Boundary Condition]
    ...
\end{definition}

\begin{definition}
    If A is a linear operator from $\mathcal{D}(A) \subset H$ to Hilbert space $H$, with an orthonormal basis of eigenfunctions $\{\phi_j\}$ 
    and corresponding increasing eigenvalues $\{\lambda_j\}$, 
    then $A^{\alpha}$ is defined as:
    \begin{equation}
        A^{\alpha}u = \sum_{j=1}^\infty \lambda_j^\alpha \langle u, \phi_j\rangle \phi_j
    \end{equation}
    and the domain $\mathcal{D}(A^{\alpha})$ is the set of all $u\in H$ such that $A^{\alpha}u\in H$.
\end{definition}


\end{theorem}

